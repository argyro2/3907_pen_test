\section{Εισαγωγή}

\hspace*{2em}\gt Στη σημερινή  εποχή, το διαδίκτυο έχει γίνει απαραίτητο για πολλές βασικές λειτουργίες που χρησιμοποιούμε στην καθημερινή μας ζωής, με τις διαδικτυακές εφαρμογές να διαδραματίζουν κρίσιμο ρόλο σε οτιδήποτε, όπως τις εφαρμογές \lt e-banking \gt ,τα μέσα κοινωνικής δικτύωσης και το ηλεκτρονικό εμπόριο. Καθώς οι οργανισμοί και οι περισσότεροι άνθρωποι βασίζονται όλο και περισσότερο σε αυτές τις εφαρμογές για να διεξάγουν τις δραστηρίοτητες αυτές και να διαχειρίζονται ευαίσθητα δεδομένα, η ανάγκη για την προστασία αυτών των εφαρμογών γίνεται πιο κρίσιμη από ποτέ. Η τεχνολογία συνεχώς εξελίσσεται, αλλά μαζί της εξελίσσονται και οι απειλές που τη στοχεύουν. Για αυτό τον λόγο ο χώρος του \lt cybersecurity \gt έχει πλέον αναδειχθεί σε ένα από τα πιο σημαντικά ζητήματα τόσο για τους προγραμματιστές όσο και για τις επιχειρήσεις και τους χρήστες.


Οι \lt web \gt εφαρμογές , από τη φύση τους, συχνά εκτίθενται στο κοινό, καθιστώντας τις πρωταρχικούς στόχους για κακόβουλους χρήστες. Αυτές οι εφαρμογές χειρίζονται συχνά μεγάλες ποσότητες ευαίσθητων δεδομένων και μπορούν να γίνουν στόχοι για εισβολείς που θέλουν να εκμεταλλευτούν αδυναμίες για κακόβουλους σκοπούς. Μεταξύ των διαφόρων μορφών επιθέσεων, η επίθεση \lt SQL injection \gt ξεχωρίζει ως μία από τις πιο διαδεδομένες και επιζήμιες. Η επίθεση αυτή περιλαμβάνει την εισαγωγή κακόβουλων \lt SQL query \gt μέσα στα πεδία εισόδου μιας \lt web \gt εφαρμογής , με αποτέλεσμα να διακινδυνεύει η βάση δεδομένων της εφαρμογήες αυτής. Αυτός ο τύπος επίθεσης μπορεί να οδηγήσει σε μη εξουσιοδοτημένη πρόσβαση σε ευαίσθητες πληροφορίες, καταστροφή δεδομένων ή ακόμα και σε πλήρη παραβίαση του συστήματος.

Οι αυξανόμενες απειλές στην ασφάλεια των συστημάτων/εφαρμογών δημιούργησαν την ανάγκη για πιο ισχυρά μέτρα ασφάλειας , γεγονός που οδήγησε στην πρακτική των έλεγχων διείσδυσης \lt(pen testing).\gt O έλεγχος διείσδυσης είναι μια προληπτική προσέγγιση για τον εντοπισμό και την αντιμετώπιση των ευάλωτων σημείων ενός συστήματος προτού μπορέσουν να τα εκμεταλλευτούν κακόβουλοι χρήστες. Με την προσομοίωση σεναρίων επιθέσεων σε πραγματικό κόσμο, οι ελεγκτές(\lt pen testers) \gtμπορούν να αποκαλύψουν αδυναμίες στην άμυνα ενός συστήματος και να παρέχουν χρήσιμες πληροφορίες για τη βελτίωση της ασφάλειας του.

Το \lt hacking \gt, που συχνά συνδέεται με παράνομες και κακόβουλες δραστηριότητες, έχει εξελιχθεί σε μια πιο διαφοροποιημένη έννοια στον τομέα του \lt cybersecurity\gt. Ενώ το ανήθικο \lt hacking \gtστοχεύει στην παραβίαση συστημάτων για προσωπικό όφελος ή για να προκαλέσει βλάβη, το ηθικό \lt hacking \gt ή αλλιώς \lt white hat hacking,\gt χρησιμεύει ως κρίσιμο εργαλείο για τη διαφύλαξη των συστημάτων πληροφοριών. Οι ηθικοί \lt hacker \gt χρησιμοποιούν τις δεξιότητές τους για να εντοπίσουν και να διορθώσουν τα ευάλωτα σημεία, διασφαλίζοντας ότι οι οργανισμοί μπορούν να προστατεύσουν τα περιουσιακά τους στοιχεία από κακόβουλες επιθέσεις. 

Καθώς οι \lt web \gt εφαρμογές  συνεχίζουν να αυξάνονται σε πολυπλοκότητα και λειτουργικότητα, η ασφάλεια τους γίνεται πιο δύσκολη και απαιτητική. Οι προγραμματιστές πρέπει όχι μόνο να επικεντρωθούν στη δημιουργία μιας αποτελεσματικής εμπειρίας χρήστη, αλλά και να διασφαλίσουν ότι οι εφαρμογές τους είναι ανθεκτικές έναντι μιας ευρείας σειράς πιθανών απειλών.Η επίτευξη αυτής της ισορροπίας απαιτεί συνεχή εκπαίδευση, ενημέρωση και χρήση προηγμένων τεχνικών ασφαλείας, ώστε να προστατεύονται οι χρήστες και τα δεδομένα τους από τις συνεχώς εξελισσόμενες απειλές. 

