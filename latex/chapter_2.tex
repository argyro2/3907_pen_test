\section{\lt Penetration testing}
\subsection{\gt Tι είναι το \lt Penetration testing}
{ 
\hspace*{2em}  Ο έλεγχος διείσδυσης , γνωστός και ως \lt penetration testing \gt ή \lt ethical hacking, \gt είναι μια διαδικασία αξιολόγησης ασφάλειας στην οποία ένας ελεγκτής προσπαθεί συστηματικά να εντοπίσει και να εκμεταλλευτεί αδυναμίες που μπορεί να παρουσιάζονται σε ένα δίκτυο, μια εφαρμογή ιστού ή σε υπολογιστικό σύστημα. Στόχος αυτής της προσομοιωμένης επίθεσης είναι να αξιολογηθεί η αποτελεσματικότητα των μέτρων ασφάλειας του συστήματος και να εντοπιστούν τυχόν αδυναμίες που θα μπορούσαν ενδεχομένως να εκμεταλλευτούν κακόβουλοι χρήστες. Το \lt penetration testing \gt περιλαμβάνει διάφορα στάδια όπως αυτά του του σχεδιασμού, της συλλογής πληροφοριών, της σάρωσης ευπάθειας, της εκμετάλλευσης και της αναφοράς.



\hspace*{1em} Κατά τη φάση σχεδιασμού, καθορίζονται το εύρος και οι στόχοι της δοκιμής, διασφαλίζοντας ότι δεν παραβιάζουν  τις πολιτικές ασφάλειας και τις απαιτήσεις συμμόρφωσης του οργανισμού. Η συλλογή πληροφοριών περιλαμβάνει τη συγκέντρωση όσων περισσότερων δεδομένων γίνεται για το σύστημα-στόχο, ώστε να εντοπιστούν πιθανά σημεία εισβολής. Η σάρωση ευπαθειών χρησιμοποιεί αυτοματοποιημένα εργαλεία για τον εντοπισμό γνωστών αδυναμιών, ενώ τεχνικές χειροκίνητων δοκιμών εφαρμόζονται για την αποκάλυψη πιο περίπλοκων ζητημάτων ασφαλείας που μπορεί να παραβλέψουν τα αυτοματοποιημένα εργαλεία.

\hspace*{1em}Μόλις εντοπιστούν οι ευπάθειες, ο ελεγκτής επιχειρεί να τις εκμεταλλευτεί για να αποκτήσει μη εξουσιοδοτημένη πρόσβαση, να αποκτήσει παραπάνω προνόμια ή να υποκλέψει ευαίσθητα δεδομένα, προσομοιώνοντας τις ενέργειες των πραγματικών εισβολέων. Στη συνέχεια, τα αποτελέσματά και οι ευπάθειες που βρέθηκαν καταγράφονται σχολαστικά σε μια λεπτομερή αναφορά, η οποία παρέχει μια ολοκληρωμένη εικόνα των αδυναμιών που βρέθηκαν, των μεθόδων που χρησιμοποιήθηκαν για την εκμετάλλευσή τους και των πιθανών επιπτώσεων στον οργανισμό.

\hspace*{1em}Η τελική φάση περιλαμβάνει την αποκατάσταση, όπου ο οργανισμός λαμβάνει συστάσεις που μπορούν να εφαρμοστούν για να διορθώσει τα ευάλωτα  σημεία και να ενισχύσει τη συνολική του ασφάλεια. Οι τακτικοί έλεγχοι διείσδυσης είναι απαραίτητοι για τη διατήρηση της καλύτερης δυνατής ασφάλειας των συστημάτων και την προστασία ευαίσθητων δεδομένων έναντι των συνεχώς εξελισσόμενων απειλών από εργαλεία \lt hacking \gt αλλά και κακόβουλων χρηστών.
} 
\subsection{\gt Τύποι \lt Penetration testing}
{ 
\hspace*{2em} Υπάρχουν πολλοί τύποι \lt penetration testing \gt , ο καθένας από αυτούς κατηγοριοποιείται ανάλογα με την προσέγγιση του. Δηλαδή η κάθε κατηγορία εξαρτάται από τις πληροφορίες που έχει ο ελεγκτής για το σύστημα  και από την τεχνική που ακολουθεί.Η κατηγοριοποίηση που ακολουθεί είναι σύμφωνα με το \lt BSI (British Standards Institution). \gt
\begin{enumerate}
    \item{\textbf{ Τύποι \lt penetration testing \gt ανάλογα με το επίπεδο γνώσης του  συστήματος από τον ελεγκτή }}
    \begin{itemize}
        \item \lt \textbf{Black Box Testing :} \gt Το \lt pen testing \gt τύπου \lt black box \gt περιλαμβάνει τον έλεγχο ενός συστήματος χωρίς καμία προηγούμενη γνώση γι' αυτό. Ο ελεγκτής ενεργεί όπως θα έκανε ένας εξωτερικός επιτιθέμενος, χωρίς πρόσβαση σε εσωτερικές πληροφορίες από το εσωτερικό του οργανισμού/εταιρείας. Αυτή η μέθοδος χρησιμοποιείται για να προσομοιώσει πραγματικά σενάρια επίθεσης και να κατανοήσει πώς μπορεί να επιτεθεί ένας εξωτερικός επιτιθέμενος την οργάνωση, ενώ εντοπίζει τυχόν ευπάθειες που μπορεί να εκμεταλλευτούν χωρίς εσωτερική γνώση. Αν και προσφέρει μια ρεαλιστική προσομοίωση εξωτερικής επίθεσης και βοηθάει στην αναγνώριση ευπαθειών που είναι δημόσια προσβάσιμες, έχει περιορισμένο εύρος, καθώς ο ελεγκτής δεν διαθέτει εσωτερική γνώση και μπορεί να μην αποκαλύψει βαθύτερες ευπάθειες μέσα στο εσωτερικό δίκτυο.
        \item \lt \textbf{White Box Testing :} \gt Το \lt pen testing \gt τύπου \lt white box περιλαμβάνει τον έλεγχο ενός συστήματος με πλήρη γνώση γι' αυτό, όπως πρόσβαση στον πηγαίο κώδικα, στην αρχιτεκτονική δικτύου και σε άλλες εσωτερικές πληροφορίες. Αυτή η προσέγγιση επιτρέπει μια λεπτομερή εξέταση των πιθανών ευπαθειών. Για παράδειγμα, μια εταιρεία ανάπτυξης λογισμικού μπορεί να πραγματοποιήσει δοκιμή white box για να διασφαλίσει ότι ο κώδικάς της είναι ασφαλής και δεν κινδυνεύει από ευπάθειες πριν την κυκλοφορία μιας νέας εφαρμογής. Αυτή η μέθοδος χρησιμοποιείται για την παροχή μιας πλήρης εκτίμησης της ασφάλειας του συστήματος  με πρόσβαση σε όλες τις σχετικές πληροφορίες και για τον εντοπισμό αλλά και την διόρθωση ευπαθειών που μπορεί να υπάρχουν κατά τη διάρκεια της διαδικασίας ανάπτυξης του συστήματος. Αν και προσφέρει μια εξαιρετικά λεπτομερή ανάλυση του συστήματος και βοηθάει στην ανίχνευση και αντιμετώπιση ευπαθειών νωρίς στη διαδικασία ανάπτυξης, απαιτεί σημαντικό χρόνο και πόρους και ενδέχεται να μην προσομοιώσει τα σενάρια πραγματικής επίθεσης όσο αποτελεσματικά όσο η τεχνική τύπου \lt black box.
    \end{itemize}
    \item{\textbf{ Τύποι \lt penetration testing \gt ανάλογα με την τεχνική}}
    \begin{itemize}
        \item \lt \textbf{Network Pen Testing :} \gtΤο \lt network pen testing \gt περιλαμβάνει την αξιολόγηση της ασφάλειας της δικτυακής υποδομής ενός οργανισμού, συμπεριλαμβανομένων των δρομολογητών, των μεταγωγέων, των \lt firewalls \gt και άλλων συσκευών δικτύου. Ο βασικός στόχος είναι να εντοπιστούν ευάλωτα σημεία που θα μπορούσαν να εκμεταλλευτούν οι εισβολείς για να αποκτήσουν μη εξουσιοδοτημένη πρόσβαση ή να διακόψουν τις υπηρεσίες δικτύου.  

       
        \item \lt \textbf{Web Application Pen Testing :} \gt  \lt web application pen testing \gt επικεντρώνεται στον εντοπισμό ευάλωτων σημείων της ασφαλείας των \lt web \gt εφαρμογών. Ο ελεγκτής προσωμειώνει επιθέσεις οι οποίες θα φανέρωναν κάποιο σημείο που θα επέτρεπε \lt sql injection,cross-site scripting (XSS) και "σπαάσιμο| της αυθεντικοποίησης.
        \item \lt \textbf{Mobile Apps Pen Testing :} \gt Το \lt mobile apps pen testing \gt αξιολογεί την ασφάλεια των εφαρμογών που εκτελούνται σε κινητές συσκευές. Αυτή η δοκιμή περιλαμβάνει την εξέταση του πηγαίου κώδικα της εφαρμογής και της αποθήκευσης δεδομένων για ευπάθειες. Ο στόχος είναι να εντοπιστούν οι αδυναμίες που θα μπορούσε να εκμεταλλευτεί ένας επιτιθέμενος για να αποκτήσει πρόσβαση σε ευαίσθητα δεδομένα ή να χειραγωγήσει τη λειτουργικότητα της εφαρμογής.
        \item \lt \textbf{Physical Pen Testing :} \gt Το \lt physical pen testing \gt αξιολογεί τους φυσικούς ελέγχους ασφαλείας ενός οργανισμού, όπως συστήματα ελέγχου πρόσβασης και επιτήρηση(\lt surveillance)\gt. Ο ελεγκτής προσπαθεί να παρακάμψει αυτούς τους ελέγχους για να αποκτήσει μη εξουσιοδοτημένη πρόσβαση σε εγκαταστάσεις και σε ευαίσθητα δεδομένα του συστήματος. 
        \item \lt \textbf{Social Engineering Pen Testing :} \gtΤο \lt social engineering pen testing \gt περιλαμβάνει προσομοίωση επιθέσεων που εκμεταλλεύονται την ανθρώπινη συμπεριφορά για να αποκτήσουν μη εξουσιοδοτημένη πρόσβαση σε πληροφορίες ή συστήματα. Ο ελεγκτής χρησιμοποιεί τεχνικές όπως  \lt phishing ,   pretexting \gt και \lt baiting \gt για να εξαπατήσει τους υπαλλήλους ωστέ να αποκαλύψουν ευαίσθητες πληροφορίες ή να εκτελέσουν ενέργειες που θέτουν σε κίνδυνο την ασφάλεια του συστήματος. 
        
    \end{itemize}
    \item{\textbf{ Τύποι \lt penetration testing \gt ανάλογα με την επιθετικότητα}}
    \begin{itemize}
        \item \textbf{\lt Passive :}\gtΤο \lt passive pen testing \gt  είναι μια μη παρεμβατική μέθοδος που συλλέγει πληροφορίες χωρίς να αλληλεπιδρά άμεσα με τα συστήματα-στόχους. Χρησιμοποιεί τεχνικές όπως \lt OSINT, network sniffing \gt και παθητική αναγνώριση για τη συλλογή δεδομένων χωρίς την εκμετάλλευση ευάλωτων σημείων. Αυτή η προσέγγιση παρέχει μια βασική κατανόηση των πιθανών ευάλωτων σημείων με βάση τις παρατηρούμενες πληροφορίες. Είναι ιδιαίτερα χρήσιμο για αρχικές φάσεις αναγνώρισης αδυναμιών ή σε εξαιρετικά ευαίσθητα περιβάλλοντα όπου η ενεργή δοκιμή είναι περιορισμένη.
        \item \textbf{\lt Cautious :}\gtΤο \lt cautious pen testing \gt δίνει προτεραιότητα στην ελαχιστοποίηση του κινδύνου για τα συστήματα-στόχους, χρησιμοποιώντας συντηρητικές τεχνικές και γνωστά \lt exploits \gt για την αποφυγή έντονων διαταραχών. Αυτή η προσέγγιση, η οποία συχνά περιορίζεται σε εύρος για τη διασφάλιση ασφάλειας, μπορεί να "χάσει" ορισμένες ευπάθειες. Είναι ιδανικό για περιβάλλοντα όπου ο χρόνος λειτουργίας και η σταθερότητα του συστήματος είναι ζωτικής σημασίας, όπως για παράδειγμα συστήματα υγειονομικής περίθαλψης.
        \item \textbf{\lt Calculated :}\gt Στο \lt calculated pen testing \gt ο ελεγκτής υπλογίζει πριν την προσωμοίωση της επίθεσης το ποσοστό επιτυχίας της αλλά και τι συνέπειες θα έχει στο σύστημα-στόχο.Είναι κατάλληλο για οργανισμούς που αναζητούν βαθιά κατανόηση της ασφαλείας τους χωρίς σημαντικούς λειτουργικούς κινδύνους.
        \item \textbf{\lt Aggressive :}\gt Το \lt aggressive pen testing \gt προσωμοιώνει τις επιθέσεις ωστέ να είναι όσο το δυνατό πιο επικίνδυνες και ρεαλιστικές γίνεται.Χρησιμοποιεί προηγμένες και επιθετικές τεχνικές, με μεγάλη πιθανότητα να προκαλέσει διακοπές λειτουργίας του συστήματος ή καταστροφή δεδομένων, για να αποκαλύψει τα πιο κρίσιμα και ευάλωτα σημεία του συστήματος.
        
    \end{itemize}
    \item{\textbf{ Τύποι \lt penetration testing \gt ανάλογα με το εύρος}}
    \begin{itemize}
        \item \textbf{\lt Full :}\gtΤο \lt full pen testing \gt είναι μια ολοκληρωμένη αξιολόγηση ολόκληρης της IT υποδομής ενός οργανισμού. Στόχος του είναι να ανακαλύψει όσο το δυνατόν περισσότερες ευπάθειες μπορεί να παρουσιάζονται σε δίκτυα, εφαρμογές, τελικά σημεία και μέτρα φυσικής ασφάλειας του οργανισμού. Αυτός ο τύπος \lt pen testing \gt προσομοιώνει μια πραγματική επίθεση με τον ελεγκτή να έχει πλήρη ελευθερία να εξερευνήσει και να εκμεταλλευτεί τυχόν αδυναμίες. 
        \item \textbf{\lt Limited :}\gtΤο \lt limited pen testing \gt εστιάζει σε συγκεκριμένα σημείεα  της IT υποδομής ενός οργανισμού, όπως συγκεκριμένες εφαρμογές, διακομιστές ή τμήματα δικτύου. Αυτή η στοχευμένη προσέγγιση στοχεύει στον εντοπισμό εύαλωτων σημείων εντός του καθορισμένου πεδίου εφαρμογής, καθιστώντας την κατάλληλη για την αντιμετώπιση γνωστών αδύναμων σημείων.

        \item \textbf{\lt Focused :} \gt Το\lt focused pen testing\gtΗ στοχεύει πολύ συγκεκριμένα ευάλωτα σημεία ή σενάρια απειλών για να ελέγξει την ανθεκτικότητα συγκεκριμένων μέτρων ασφαλείας. Συχνά ο ελεγκτής είναι καθοδηγούμενος από ήδη γνωστά ζητήματα ή πρόσφατες πληροφορίες απειλών, Αυτή η συγκεκριμένη προσέγγιση επιβεβαιώνει πόσο καλά λειτουργούν συγκεκριμένες άμυνες ή εξετάζει τις επιπτώσεις από συγκεκριμένα ευάλωτα σημεία. 
    \end{itemize}
    \item{\textbf{ Τύποι \lt penetration testing \gt ανάλογα με την προσέγγιση}}\
    \begin{itemize}
        \item \textbf{\lt Covert :} \gt Σε ενά  \lt covert pen testing \gt γνωστό και ως \lt double-blind pen testing \gt η επίθεση στο σύστημα γίνεται από κάποιο ελεγκτή χωρίς να έχουν ενημερωθεί οι υπάλληλοι υπεύθυνοι για την ασφάλεια  του συστήματος. Αυτή η προσέγγιση στοχεύει στην προσομοίωση ενός πραγματικού σεναρίου επίθεσης, όπου ο ελεγκτής συμπεριφέρεται σαν αληθινός εισβολέας  δηλαδή δεν έχει προηγούμενη γνώση της εσωτερικής υποδομής του συστήματος και πρέπει να βασίζεται σε τεχνικές εξωτερικής αναγνώρισης και εκμετάλλευσης.
        \item \textbf{\lt Overt :}Το \lt overt pen testing \gt περιλαμβάνει τη διεξαγωγή των επιθέσεων με την πλήρη γνώση και συνεργασία της ομάδας των υπαλλήλων υπεύθυνους για την ασφάλεια του οργανισμού. Ο ελεγκτής παρέχεται με λεπτομερείς πληροφορίες σχετικά με εσωτερικά συστήματα, δίκτυα και εφαρμογές. 
        
    \end{itemize}
    \item{\textbf{ Τύποι \lt penetration testing \gt ανάλογα με το σημείο εκκίνησης}}
    \begin{itemize}
        \item \textbf{\lt Internal :} \gt Ο έλεγχος τύπου \lt internal  testing \gt περιλαμβάνει την προσομοίωση επιθέσεων από το εσωτερικό του δικτύου του οργανισμού. Αυτός ο τύπος \lt pen testing \gt είναι απαραίτητος για την κατανόηση του τρόπου με τον οποίο ένας εσωτερικός χρήστης, όπως ένας υπάλληλος ή κάποιος που έχει καταφέρει να παραβιάσει την εξωτερική ασφάλεια του δικτύου, θα μπορούσε να εκμεταλλευτεί τυχόν ευπάθειες που υπάροχυν στο σύστημα. Για παράδειγμα, μια βιομηχανική επιχείρηση μπορεί να πραγματοποιήσει \lt internal pen testing \gt για να διασφαλίσει ότι οι υπάλληλοι δεν μπορούν εύκολα να αποκτήσουν πρόσβαση σε ευαίσθητα τμήματα του δικτύου ή να διαταράξουν με κάποιο τρόπο το σύστημα.
        \item \textbf{\lt External :} \gt Ο έλεγχος τύπου \lt external  testing \gt  επικεντρώνεται στην αξιολόγηση της ασφάλειας των εξωτερικών πόρων ενός οργανισμού, όπως διακομιστές, \lt firewalls \gt και άλλες υποδομές δικτύου που είναι εκτεθειμένες στο διαδίκτυο. Αυτού του είδους η δοκιμή είναι κρίσιμη για την αναγνώριση ευπαθειών που θα μπορούσαν να εκμεταλλευτούν επιτιθέμενοι εκτός του εσωτερικού δικτύου του οργανισμού. Για παράδειγμα, μια  εταιρεία που προσφέρει υπηρεσίες \lt online banking \gt  θα επέλεγε  \lt external pen testing \gt  για να διασφαλίσει ότι η πλατφόρμα της, οι διακομιστές web και η σχετική υποδομή είναι ασφαλείς από επιθέσεις όπως \lt DDoS, SQL injection \gt και μη εξουσιοδοτημένη πρόσβαση.
    \end{itemize}
\end{enumerate}
}
\subsection{\gt Γνωστές Μεθοδολογίες του \lt Penetration testing}
\hspace*{2em}Καθώς το \lt penetration testing \gtείναι πολύ σημαντικό για την ασφάλεια συστημάτων είναι απαραίτητο να υπάρχει κάποια μεθοδολογία για την διεξαγωγή του. Γνωστές μεθοδολογίες οι οποίες παρέχουν λεπτομερείς οδηγίες και βέλτιστες πρακτικές για τη διεξαγωγή ολοκληρωμένων \lt pen test \gt είναι οι \lt OWASP Testing Guide (OTG), Penetration Testing Execution Standard (PTES),NIST SP 800-115 και \lt OSSTMM (Open Source Security Testing Methodology Manual).\gt
\subsection{\gt Εργαλεία και Τεχνικές του \lt Penetration testing}
