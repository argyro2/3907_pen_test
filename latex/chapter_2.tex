\section{Θεωρητικό Κομμάτι}
\subsection{\lt Penetration Testing}
\subsubsection{\gt Tι είναι το \lt Penetration testing}
{ 
\hspace*{2em}  Ο έλεγχος διείσδυσης , γνωστός και ως \lt penetration testing \gt ή \lt ethical hacking, \gt είναι μια διαδικασία αξιολόγησης ασφάλειας στην οποία ένας ελεγκτής προσπαθεί συστηματικά να εντοπίσει και να εκμεταλλευτεί αδυναμίες που μπορεί να παρουσιάζονται σε ένα δίκτυο, μια εφαρμογή ιστού ή σε υπολογιστικό σύστημα. Στόχος αυτής της προσομοιωμένης επίθεσης είναι να αξιολογηθεί η αποτελεσματικότητα των μέτρων ασφάλειας του συστήματος και να εντοπιστούν τυχόν αδυναμίες που θα μπορούσαν ενδεχομένως να εκμεταλλευτούν κακόβουλοι χρήστες. Το \lt penetration testing \gt περιλαμβάνει διάφορα στάδια όπως αυτά του του σχεδιασμού, της συλλογής πληροφοριών, της σάρωσης ευπάθειας, της εκμετάλλευσης και της αναφοράς.



\hspace*{1em} Κατά τη φάση σχεδιασμού, καθορίζονται το εύρος και οι στόχοι της δοκιμής, διασφαλίζοντας ότι δεν παραβιάζουν  τις πολιτικές ασφάλειας και τις απαιτήσεις συμμόρφωσης του οργανισμού. Η συλλογή πληροφοριών περιλαμβάνει τη συγκέντρωση όσων περισσότερων δεδομένων γίνεται για το σύστημα-στόχο, ώστε να εντοπιστούν πιθανά σημεία εισβολής. Η σάρωση ευπαθειών χρησιμοποιεί αυτοματοποιημένα εργαλεία για τον εντοπισμό γνωστών αδυναμιών, ενώ τεχνικές χειροκίνητων δοκιμών εφαρμόζονται για την αποκάλυψη πιο περίπλοκων ζητημάτων ασφαλείας που μπορεί να παραβλέψουν τα αυτοματοποιημένα εργαλεία.

\hspace*{1em}Μόλις εντοπιστούν οι ευπάθειες, ο ελεγκτής επιχειρεί να τις εκμεταλλευτεί για να αποκτήσει μη εξουσιοδοτημένη πρόσβαση, να αποκτήσει παραπάνω προνόμια ή να υποκλέψει ευαίσθητα δεδομένα, προσομοιώνοντας τις ενέργειες των πραγματικών εισβολέων. Στη συνέχεια, τα αποτελέσματά και οι ευπάθειες που βρέθηκαν καταγράφονται σχολαστικά σε μια λεπτομερή αναφορά, η οποία παρέχει μια ολοκληρωμένη εικόνα των αδυναμιών που βρέθηκαν, των μεθόδων που χρησιμοποιήθηκαν για την εκμετάλλευσή τους και των πιθανών επιπτώσεων στον οργανισμό.

\hspace*{1em}Η τελική φάση περιλαμβάνει την αποκατάσταση, όπου ο οργανισμός λαμβάνει συστάσεις που μπορούν να εφαρμοστούν για να διορθώσει τα ευάλωτα  σημεία και να ενισχύσει τη συνολική του ασφάλεια. Οι τακτικοί έλεγχοι διείσδυσης είναι απαραίτητοι για τη διατήρηση της καλύτερης δυνατής ασφάλειας των συστημάτων και την προστασία ευαίσθητων δεδομένων έναντι των συνεχώς εξελισσόμενων απειλών από εργαλεία \lt hacking \gt αλλά και κακόβουλων χρηστών.
} 
\subsubsection{\gt Τύποι \lt Penetration testing}
{ 
\hspace*{2em} Υπάρχουν πολλοί τύποι \lt penetration testing \gt , ο καθένας από αυτούς κατηγοριοποιείται ανάλογα με την προσέγγιση του. Δηλαδή η κάθε κατηγορία εξαρτάται από τις πληροφορίες που έχει ο ελεγκτής για το σύστημα  και από την τεχνική που ακολουθεί.Η κατηγοριοποίηση που ακολουθεί είναι σύμφωνα με το \lt BSI (British Standards Institution). \gt
\begin{enumerate}
    \item{\textbf{ Τύποι \lt penetration testing \gt ανάλογα με το επίπεδο γνώσης του  συστήματος από τον ελεγκτή }}
    \begin{itemize}
        \item \lt \textbf{Black Box Testing :} \gt Το \lt pen testing \gt τύπου \lt black box \gt περιλαμβάνει τον έλεγχο ενός συστήματος χωρίς καμία προηγούμενη γνώση γι' αυτό. Ο ελεγκτής ενεργεί όπως θα έκανε ένας εξωτερικός επιτιθέμενος, χωρίς πρόσβαση σε εσωτερικές πληροφορίες από το εσωτερικό του οργανισμού/εταιρείας. Αυτή η μέθοδος χρησιμοποιείται για να προσομοιώσει πραγματικά σενάρια επίθεσης και να κατανοήσει πώς μπορεί να επιτεθεί ένας εξωτερικός επιτιθέμενος την οργάνωση, ενώ εντοπίζει τυχόν ευπάθειες που μπορεί να εκμεταλλευτούν χωρίς εσωτερική γνώση. Αν και προσφέρει μια ρεαλιστική προσομοίωση εξωτερικής επίθεσης και βοηθάει στην αναγνώριση ευπαθειών που είναι δημόσια προσβάσιμες, έχει περιορισμένο εύρος, καθώς ο ελεγκτής δεν διαθέτει εσωτερική γνώση και μπορεί να μην αποκαλύψει βαθύτερες ευπάθειες μέσα στο εσωτερικό δίκτυο.
        \item \lt \textbf{White Box Testing :} \gt Το \lt pen testing \gt τύπου \lt white box περιλαμβάνει τον έλεγχο ενός συστήματος με πλήρη γνώση γι' αυτό, όπως πρόσβαση στον πηγαίο κώδικα, στην αρχιτεκτονική δικτύου και σε άλλες εσωτερικές πληροφορίες. Αυτή η προσέγγιση επιτρέπει μια λεπτομερή εξέταση των πιθανών ευπαθειών. Για παράδειγμα, μια εταιρεία ανάπτυξης λογισμικού μπορεί να πραγματοποιήσει δοκιμή white box για να διασφαλίσει ότι ο κώδικάς της είναι ασφαλής και δεν κινδυνεύει από ευπάθειες πριν την κυκλοφορία μιας νέας εφαρμογής. Αυτή η μέθοδος χρησιμοποιείται για την παροχή μιας πλήρης εκτίμησης της ασφάλειας του συστήματος  με πρόσβαση σε όλες τις σχετικές πληροφορίες και για τον εντοπισμό αλλά και την διόρθωση ευπαθειών που μπορεί να υπάρχουν κατά τη διάρκεια της διαδικασίας ανάπτυξης του συστήματος. Αν και προσφέρει μια εξαιρετικά λεπτομερή ανάλυση του συστήματος και βοηθάει στην ανίχνευση και αντιμετώπιση ευπαθειών νωρίς στη διαδικασία ανάπτυξης, απαιτεί σημαντικό χρόνο και πόρους και ενδέχεται να μην προσομοιώσει τα σενάρια πραγματικής επίθεσης όσο αποτελεσματικά όσο η τεχνική τύπου \lt black box.
    \end{itemize}
    \item{\textbf{ Τύποι \lt penetration testing \gt ανάλογα με την τεχνική}}
    \begin{itemize}
        \item \lt \textbf{Network Pen Testing :} \gtΤο \lt network pen testing \gt περιλαμβάνει την αξιολόγηση της ασφάλειας της δικτυακής υποδομής ενός οργανισμού, συμπεριλαμβανομένων των δρομολογητών, των μεταγωγέων, των \lt firewalls \gt και άλλων συσκευών δικτύου. Ο βασικός στόχος είναι να εντοπιστούν ευάλωτα σημεία που θα μπορούσαν να εκμεταλλευτούν οι εισβολείς για να αποκτήσουν μη εξουσιοδοτημένη πρόσβαση ή να διακόψουν τις υπηρεσίες δικτύου.  

       
        \item \lt \textbf{Web Application Pen Testing :} \gt  \lt web application pen testing \gt επικεντρώνεται στον εντοπισμό ευάλωτων σημείων της ασφαλείας των \lt web \gt εφαρμογών. Ο ελεγκτής προσωμειώνει επιθέσεις οι οποίες θα φανέρωναν κάποιο σημείο που θα επέτρεπε \lt sql injection,cross-site scripting (XSS) και "σπάσιμο" της αυθεντικοποίησης.
        \item \lt \textbf{Mobile Apps Pen Testing :} \gt Το \lt mobile apps pen testing \gt αξιολογεί την ασφάλεια των εφαρμογών που εκτελούνται σε κινητές συσκευές. Αυτή η δοκιμή περιλαμβάνει την εξέταση του πηγαίου κώδικα της εφαρμογής και της αποθήκευσης δεδομένων για ευπάθειες. Ο στόχος είναι να εντοπιστούν οι αδυναμίες που θα μπορούσε να εκμεταλλευτεί ένας επιτιθέμενος για να αποκτήσει πρόσβαση σε ευαίσθητα δεδομένα ή να χειραγωγήσει τη λειτουργικότητα της εφαρμογής.
        \item \lt \textbf{Physical Pen Testing :} \gt Το \lt physical pen testing \gt αξιολογεί τους φυσικούς ελέγχους ασφαλείας ενός οργανισμού, όπως συστήματα ελέγχου πρόσβασης και επιτήρηση(\lt surveillance)\gt. Ο ελεγκτής προσπαθεί να παρακάμψει αυτούς τους ελέγχους για να αποκτήσει μη εξουσιοδοτημένη πρόσβαση σε εγκαταστάσεις και σε ευαίσθητα δεδομένα του συστήματος. 
        \item \lt \textbf{Social Engineering Pen Testing :} \gtΤο \lt social engineering pen testing \gt περιλαμβάνει προσομοίωση επιθέσεων που εκμεταλλεύονται την ανθρώπινη συμπεριφορά για να αποκτήσουν μη εξουσιοδοτημένη πρόσβαση σε πληροφορίες ή συστήματα. Ο ελεγκτής χρησιμοποιεί τεχνικές όπως  \lt phishing ,   pretexting \gt και \lt baiting \gt για να εξαπατήσει τους υπαλλήλους ωστέ να αποκαλύψουν ευαίσθητες πληροφορίες ή να εκτελέσουν ενέργειες που θέτουν σε κίνδυνο την ασφάλεια του συστήματος. 
        
    \end{itemize}
    \item{\textbf{ Τύποι \lt penetration testing \gt ανάλογα με την επιθετικότητα}}
    \begin{itemize}
        \item \textbf{\lt Passive :}\gtΤο \lt passive pen testing \gt  είναι μια μη παρεμβατική μέθοδος που συλλέγει πληροφορίες χωρίς να αλληλεπιδρά άμεσα με τα συστήματα-στόχους. Χρησιμοποιεί τεχνικές όπως \lt OSINT, network sniffing \gt και παθητική αναγνώριση για τη συλλογή δεδομένων χωρίς την εκμετάλλευση ευάλωτων σημείων. Αυτή η προσέγγιση παρέχει μια βασική κατανόηση των πιθανών ευάλωτων σημείων με βάση τις παρατηρούμενες πληροφορίες. Είναι ιδιαίτερα χρήσιμο για αρχικές φάσεις αναγνώρισης αδυναμιών ή σε εξαιρετικά ευαίσθητα περιβάλλοντα όπου η ενεργή δοκιμή είναι περιορισμένη.
        \item \textbf{\lt Cautious :}\gtΤο \lt cautious pen testing \gt δίνει προτεραιότητα στην ελαχιστοποίηση του κινδύνου για τα συστήματα-στόχους, χρησιμοποιώντας συντηρητικές τεχνικές και γνωστά \lt exploits \gt για την αποφυγή έντονων διαταραχών. Αυτή η προσέγγιση, η οποία συχνά περιορίζεται σε εύρος για τη διασφάλιση ασφάλειας, μπορεί να "χάσει" ορισμένες ευπάθειες. Είναι ιδανικό για περιβάλλοντα όπου ο χρόνος λειτουργίας και η σταθερότητα του συστήματος είναι ζωτικής σημασίας, όπως για παράδειγμα συστήματα υγειονομικής περίθαλψης.
        \item \textbf{\lt Calculated :}\gt Στο \lt calculated pen testing \gt ο ελεγκτής υπλογίζει πριν την προσωμοίωση της επίθεσης το ποσοστό επιτυχίας της αλλά και τι συνέπειες θα έχει στο σύστημα-στόχο.Είναι κατάλληλο για οργανισμούς που αναζητούν βαθιά κατανόηση της ασφαλείας τους χωρίς σημαντικούς λειτουργικούς κινδύνους.
        \item \textbf{\lt Aggressive :}\gt Το \lt aggressive pen testing \gt προσωμοιώνει τις επιθέσεις ωστέ να είναι όσο το δυνατό πιο επικίνδυνες και ρεαλιστικές γίνεται.Χρησιμοποιεί προηγμένες και επιθετικές τεχνικές, με μεγάλη πιθανότητα να προκαλέσει διακοπές λειτουργίας του συστήματος ή καταστροφή δεδομένων, για να αποκαλύψει τα πιο κρίσιμα και ευάλωτα σημεία του συστήματος.
        
    \end{itemize}
    \item{\textbf{ Τύποι \lt penetration testing \gt ανάλογα με το εύρος}}
    \begin{itemize}
        \item \textbf{\lt Full :}\gtΤο \lt full pen testing \gt είναι μια ολοκληρωμένη αξιολόγηση ολόκληρης της IT υποδομής ενός οργανισμού. Στόχος του είναι να ανακαλύψει όσο το δυνατόν περισσότερες ευπάθειες μπορεί να παρουσιάζονται σε δίκτυα, εφαρμογές, τελικά σημεία και μέτρα φυσικής ασφάλειας του οργανισμού. Αυτός ο τύπος \lt pen testing \gt προσομοιώνει μια πραγματική επίθεση με τον ελεγκτή να έχει πλήρη ελευθερία να εξερευνήσει και να εκμεταλλευτεί τυχόν αδυναμίες. 
        \item \textbf{\lt Limited :}\gtΤο \lt limited pen testing \gt εστιάζει σε συγκεκριμένα σημείεα  της IT υποδομής ενός οργανισμού, όπως συγκεκριμένες εφαρμογές, διακομιστές ή τμήματα δικτύου. Αυτή η στοχευμένη προσέγγιση στοχεύει στον εντοπισμό εύαλωτων σημείων εντός του καθορισμένου πεδίου εφαρμογής, καθιστώντας την κατάλληλη για την αντιμετώπιση γνωστών αδύναμων σημείων.

        \item \textbf{\lt Focused :} \gt Το\lt focused pen testing\gtΗ στοχεύει πολύ συγκεκριμένα ευάλωτα σημεία ή σενάρια απειλών για να ελέγξει την ανθεκτικότητα συγκεκριμένων μέτρων ασφαλείας. Συχνά ο ελεγκτής είναι καθοδηγούμενος από ήδη γνωστά ζητήματα ή πρόσφατες πληροφορίες απειλών, Αυτή η συγκεκριμένη προσέγγιση επιβεβαιώνει πόσο καλά λειτουργούν συγκεκριμένες άμυνες ή εξετάζει τις επιπτώσεις από συγκεκριμένα ευάλωτα σημεία. 
    \end{itemize}
    \item{\textbf{ Τύποι \lt penetration testing \gt ανάλογα με την προσέγγιση}}\
    \begin{itemize}
        \item \textbf{\lt Covert :} \gt Σε ενά  \lt covert pen testing \gt γνωστό και ως \lt double-blind pen testing \gt η επίθεση στο σύστημα γίνεται από κάποιο ελεγκτή χωρίς να έχουν ενημερωθεί οι υπάλληλοι υπεύθυνοι για την ασφάλεια  του συστήματος. Αυτή η προσέγγιση στοχεύει στην προσομοίωση ενός πραγματικού σεναρίου επίθεσης, όπου ο ελεγκτής συμπεριφέρεται σαν αληθινός εισβολέας  δηλαδή δεν έχει προηγούμενη γνώση της εσωτερικής υποδομής του συστήματος και πρέπει να βασίζεται σε τεχνικές εξωτερικής αναγνώρισης και εκμετάλλευσης.
        \item \textbf{\lt Overt :}Το \lt overt pen testing \gt περιλαμβάνει τη διεξαγωγή των επιθέσεων με την πλήρη γνώση και συνεργασία της ομάδας των υπαλλήλων υπεύθυνους για την ασφάλεια του οργανισμού. Ο ελεγκτής παρέχεται με λεπτομερείς πληροφορίες σχετικά με εσωτερικά συστήματα, δίκτυα και εφαρμογές. 
        
    \end{itemize}
    \item{\textbf{ Τύποι \lt penetration testing \gt ανάλογα με το σημείο εκκίνησης}}
    \begin{itemize}
        \item \textbf{\lt Internal :} \gt Ο έλεγχος τύπου \lt internal  testing \gt περιλαμβάνει την προσομοίωση επιθέσεων από το εσωτερικό του δικτύου του οργανισμού. Αυτός ο τύπος \lt pen testing \gt είναι απαραίτητος για την κατανόηση του τρόπου με τον οποίο ένας εσωτερικός χρήστης, όπως ένας υπάλληλος ή κάποιος που έχει καταφέρει να παραβιάσει την εξωτερική ασφάλεια του δικτύου, θα μπορούσε να εκμεταλλευτεί τυχόν ευπάθειες που υπάροχυν στο σύστημα. Για παράδειγμα, μια βιομηχανική επιχείρηση μπορεί να πραγματοποιήσει \lt internal pen testing \gt για να διασφαλίσει ότι οι υπάλληλοι δεν μπορούν εύκολα να αποκτήσουν πρόσβαση σε ευαίσθητα τμήματα του δικτύου ή να διαταράξουν με κάποιο τρόπο το σύστημα.
        \item \textbf{\lt External :} \gt Ο έλεγχος τύπου \lt external  testing \gt  επικεντρώνεται στην αξιολόγηση της ασφάλειας των εξωτερικών πόρων ενός οργανισμού, όπως διακομιστές, \lt firewalls \gt και άλλες υποδομές δικτύου που είναι εκτεθειμένες στο διαδίκτυο. Αυτού του είδους η δοκιμή είναι κρίσιμη για την αναγνώριση ευπαθειών που θα μπορούσαν να εκμεταλλευτούν επιτιθέμενοι εκτός του εσωτερικού δικτύου του οργανισμού. Για παράδειγμα, μια  εταιρεία που προσφέρει υπηρεσίες \lt online banking \gt  θα επέλεγε  \lt external pen testing \gt  για να διασφαλίσει ότι η πλατφόρμα της, οι διακομιστές web και η σχετική υποδομή είναι ασφαλείς από επιθέσεις όπως \lt DDoS, SQL injection \gt και μη εξουσιοδοτημένη πρόσβαση.
    \end{itemize}
\end{enumerate}
}
\subsubsection{\gt Γνωστές Μεθοδολογίες του \lt Penetration testing}
\hspace*{2em}Καθώς το \lt penetration testing \gtείναι πολύ σημαντικό για την ασφάλεια συστημάτων είναι απαραίτητο να υπάρχει κάποια μεθοδολογία για την διεξαγωγή του. Γνωστές μεθοδολογίες οι οποίες παρέχουν λεπτομερείς οδηγίες και βέλτιστες πρακτικές για τη διεξαγωγή ολοκληρωμένων \lt pen test \gt είναι οι \lt OWASP Testing Guide (OTG), Penetration Testing Execution Standard (PTES),NIST SP 800-115 και \lt OSSTMM (Open Source Security Testing Methodology Manual).Παρακάτω θα αναλυθούν οι φάσεις από κάποιες από αυτές τις μεθοδολογίες.

Η μεθοδολογία \lt Penetration Testing Execution Standard (PTES) \gt παρέχει ένα ολοκληρωμένο πλαίσιο για τη διεξαγωγή \lt pen testing\gt, διασφαλίζοντας ότι όλα τα βήματα της διαδικασίας εκτελούνται μεθοδικά και σωστά.Αυτή η ενότητα θα αναλύσει τις επτά διαφορετικές φάσεις του \lt PTES, \gt καθεμία από τις οποίες είναι απαραίτητη για τον εντοπισμό και την αντιμετώπιση των αδυναμιών ασφαλείας στην υποδομή ενός οργανισμού.

\begin{enumerate}
    \item \textbf{\lt  Pre-Engagement}

    
   Η αρχική φάση περιλαμβάνει την προετοιμασία και τον σχεδιασμό της επίθεσης που πρόκειται να προσομοιωθεί.Οι ελεγκτές συγκεντρώνουν τα κατάλληλα εργαλεία, λειτουργικά συστήματα και λογισμικό που χρειάζονται, ανάλογα με το είδος και την έκταση του \lt pen test\gt. 

    \item \textbf{\lt  Information Gathering}

Σε αυτή τη φάση, ο ελεγκτής συγκεντρώνει πληροφορίες για τα συστήματα-στόχους, τόσο από τον οργανισμό όσο και από δημόσιες πηγές, όπως τα μέσα κοινωνικής δικτύωσης. Αυτή η συλλογή πληροφοριών είναι σημαντική για τα \lt network pen testing \gt καθώς βοηθά στον εντοπισμό πιθανών αδυναμιών και σημείων εισόδου.

\item \textbf{\lt  Threat Modelling}

Η μοντελοποίηση απειλών(\lt threat modelling\gt) είναι μια συστηματική προσέγγιση για τον εντοπισμό και την αντιμετώπιση πιθανών απειλών που στοχεύουν ένα σύστημα ή μια εφαρμογή ενός οργανισμού. Αυτή η φάση περιλαμβάνει την αξιολόγηση των δυνατοτήτων των πιθανών εισβολέων και τον σχεδιασμό της κατάλληλης ασφάλειας για την προστασία των πιο ευάλωτων τμημάτων του οργανισμού.


\item \textbf{\lt  Vulnerability Analysis}

Σε αυτή την φάση  οι ελεγκτές εντοπίζουν και αξιολογούν τα ευάλωτα σημεία που προκύπτουν από ελαττώματα/αδυναμίες στην ασφάλεια των συστημάτων  του οργανισμού. Χρησιμοποιούν έναν συνδυασμό αυτοματοποιημένων εργαλείων και χειροκίνητων μεθόδων για να αποκαλύψουν αδυναμίες που θα μπορούσαν να εκμεταλλευτούν κακόβουλοι χρήστες. Ο στόχος είναι να εντοπιστούν τα πιο επικίνδυνα ευάλωτα σημεία και να δοθεί στον οργανισμό μια ξεκάθαρη εικόνα των περιοχών που χρειάζονται άμεση προσοχή και διόρθωση.

\item \textbf{\lt  Exploitation}

Σε αυτή την φάση οι ελεγκτές δοκιμάζουν να εκμεταλλευτούν τα ευάλωτα σημεία που βρέθηκαν στις προηγούμενες φάσεις για να δουν αν μπορούν να χρησιμοποιηθούν ώστε να παραβιαστεί η ασφάλεια του οργανισμού.Αυτό περιλαμβάνει την προσομοίωση πραγματικών επιθέσεων για να ελέγξουν αν οι αδυναμίες που εντοπίστηκαν μπορούν να αξιοποιηθούν. Η επιτυχία αυτής της φάσης εξαρτάται από την ακρίβεια της προηγούμενης ανάλυσης και δείχνει τα αποτελέσματα που θα είχε  μια πραγματική επίθεση.

\item \textbf{\lt Post Exploitation}


Μετά την επιτυχή εκμετάλλευση των αδυναμιών, ο ελεγκτής αξιολογεί το επίπεδο πρόσβασης που απέκτησε και την αξία των παραβιασμένων συστημάτων. Αυτή η φάση δείχνει πόσο μακριά θα μπορούσε να φτάσει ένας εισβολέας μέσα στο δίκτυο και ποιος θα ήταν ο αντίκτυπος στη συνολική ασφάλεια του οργανισμού. Ο ελεγκτής επίσης διορθώνει τυχόν αλλαγές που έγιναν κατά τη διάρκεια της δοκιμής για να διασφαλίσει ότι δεν θα μείνουν πίσω ανεπιθύμητες επιπτώσεις.

\item \textbf{\lt Reporting}

Η τελική φάση περιλαμβάνει τη συγκέντρωση των αδυναμιών που βρέθηκαν και τις επιπτώσεις που θα είχαν αν συνέβαιναν σε μια ολοκληρωμένη αναφορά. Στην αναφορά αυτή σημειώνεται τι δοκιμάστηκε, ποιες μέθοδοι χρησιμοποιήθηκαν, τα ευάλωτα σημεία που βρέθηκαν και τα βήματα που έγιναν για την εκμετάλλευσή τους. Παρέχει επίσης προτάσεις για τη μείωση των κινδύνων, βοηθώντας τον οργανισμό να βελτιώσει την ασφάλειά του. 

\end{enumerate}


\begin{tikzpicture}[
    node distance=2cm,
    every node/.style={rectangle, draw, minimum width=2.5cm, minimum height=1cm, align=center},
    >=Stealth
]
\lt
% Nodes
\node (pre-engagement) {Pre-engagement};
\node (info-gathering) [right=of pre-engagement] {Information\\Gathering};
\node (threat-modelling) [right=of info-gathering] {Threat Modelling};
\node (vulnerability-analysis) [below=of pre-engagement] {Vulnerability\\Analysis};
\node (exploitation) [right=of vulnerability-analysis] {Exploitation};
\node (post-exploitation) [right=of exploitation] {Post Exploitation};
\node (reporting) [below=of vulnerability-analysis] {Reporting};

% Arrows
\draw[->] (pre-engagement) -- (info-gathering);
\draw[->] (info-gathering) -- (threat-modelling);
\draw[->] (threat-modelling) -- (vulnerability-analysis);
\draw[->] (vulnerability-analysis) -- (exploitation);
\draw[->] (exploitation) -- (post-exploitation);
\draw[->] (post-exploitation) -- (reporting);



\end{tikzpicture}
\lt
\begin{center}
    \textbf{ Penetration Testing Execution Standard (PTES) Phases}
\end{center}




Το Εθνικό Ινστιτούτο Προτύπων και Τεχνολογίας \lt (NIST) \gt παρέχει μια δομημένη προσέγγιση για την διεξαγωγή \lt pen testing\gt, σχεδιασμένη να βοηθά τους οργανισμούς να αξιολογούν την ασφάλεια των συστημάτων  τους. Η μεθοδολογία \lt NIST \gt χωρίζεται σε τέσσερις βασικές φάσεις:\lt Planning, Discovery, Attack \gt και \lt Reporting.\gt Θα αναλυθούν παρακάτω.

\begin{enumerate}
    \item \textbf{\lt Planning}

    
Η πρώτη φάση είναι το θεμελιώδες βήμα της μεθοδολογίας  NIST. Σε αυτή τη φάση, η ομάδα που θα κάνει τον έλεγχο και ο οργανισμός συμφωνούν για το τι ακριβώς θα δοκιμαστεί, όπως ποια συστήματα, δίκτυα και εφαρμογές θα ελεγχθούν, και ποια όρια θα πρέπει να τηρηθούν. Επίσης, καθορίζονται και άλλοι παράμετροι όπως  ο χρόνος που θα διαρκέσει η δοκιμή, οι πιθανοί κίνδυνοι που μπορεί να εμφανιστούν και τα εργαλεία ή το προσωπικό που θα χρειαστεί.

    \item \textbf{\lt Discovery}

    
Στην φάση αυτή οι ελεγκτές επικεντρώνεται στη συλλογή όσο το δυνατόν περισσότερων πληροφοριών σχετικά με το σύστημα-στόχο. Η συλλογή πληροφοριών γίνεται  τόσο με παθητικές τεχνικές, όπως τον έλεγχο ήδη δημόσιων πληροφοριών και σάρωση δικτύου, όσο και με ενεργητικές τεχνικές, όπως συστήματα ανίχνευσης για ανοιχτές θύρες, υπηρεσίες και άλλα εκμεταλλεύσιμα σημεία εισόδου.

    \item \textbf{\lt Attack}

    
Στη φάση της επίθεσης, ο ελεγκτής χρησιμοποιεί τις πληροφορίες που συγκεντρώθηκαν κατά τη διάρκεια του \lt Discovery \gt για να προσπαθήσει να εκμεταλλευτεί τα ευάλωτα σημεία που βρέθηκαν. Αυτή η φάση προσομοιώνει τις ενέργειες ενός πραγματικού εισβολέα, χρησιμοποιώντας διάφορα εργαλεία και τεχνικές για να παραβιάσει τις άμυνες του οργανισμού. Ο στόχος είναι να αποκτήσει μη εξουσιοδοτημένη πρόσβαση σε συστήματα, δεδομένα ή δίκτυα για να δείξει τον πιθανό αντίκτυπο μιας επιτυχημένης επίθεσης. Η φάση αυτή είναι συχνά το πιο σημαντικό μέρος ενός \lt pen test \gt, καθώς παρέχει  αποδείξεις για τα κενά ασφαλείας και την πιθανή εκμετάλλευσή τους από κάποιον κακόβουλο χρήστη.
    \item \textbf{\lt Reporting}

    Στην τελευταία φάση  συντάσσεται μια αναλυτική αναφορά με τις αδυναμίες που ανακαλύγθηκαν και τα αποτελέσματά της εκμετάλλευσης τους. Η αναφορά περιλαμβάνει συνήθως μια  περίληψη για τα ανώτερα  του οργανισμού, που περιγράφει τη συνολική κατάσταση της ασφαλείας και τα βασικά ευάλωτα σημεία, καθώς και μια πιο τεχνική ενότητα που παρέχει σε βάθος ανάλυση των συγκεκριμένων ευρημάτων.Επίσης περιέχει και προτάσεις που θα βοηθήσουν στον μετριασμό των κινδύνων που εντοπίστηκαν, βοηθώντας τον οργανισμό να βελτιώσει την ασφάλεια των συστημάτων του.
\end{enumerate}


\begin{tikzpicture}[
    node distance=2cm,
    every node/.style={rectangle, draw, minimum width=2.5cm, minimum height=1cm, align=center},
    >=Stealth
]
\lt
% Nodes
\node (planning) {Planning};
\node (discovery) [right=of planning] {Discovery};
\node (attack) [right=of discovery] {Attack};
\node(reporting)[below=of planning] {Reporting};

% Arrows
\draw[->] (planning) -- (discovery);
\draw[->] (discovery) -- (attack);
\draw[->] (attack) -- (reporting);
\draw[->] (attack.north) .. controls +(up:1cm) and +(up:1cm) .. node[above] {Additional Discovery} (discovery.north);



\end{tikzpicture}
\begin{center}
    \textbf{\lt NIST SP 800-115 Phases}
\end{center}


Η μεθοδολογία \lt OWASP (OTG) \gt προσφέρει έναν οργανωμένο τρόπο δειξαγωγής \lt pen testing \gt, περιλαμβάνοντας βασικά στάδια όπως  συλλογή πληροφοριών, δοκιμές ρυθμίσεων και  αξιολόγηση ευπαθειών. Στόχος της είναι να εντοπίσει και να μειώσει πιθανούς κινδύνους ασφαλείας σε κάθε φάση της ανάπτυξης λογισμικού. Τέλος η μεθοδολογία \lt Open Source Security Testing Methodology Manual (OSSTMM) \gt παρέχει μια πιο ευρεία προσέγγιση, καλύπτοντας όχι μόνο web εφαρμογές , αλλά και άλλους τομείς της ασφάλειας, όπως τη φυσική και δικτυακή ασφάλεια. Δημιουργήθηκε από το \lt Institute for Security and Open Methodologies (ISECOM) \gtκαι δίνει έμφαση σε μια αυστηρή διαδικασία ελέγχου με λεπτομερείς οδηγίες για την αξιολόγηση της ασφάλειας σε διάφορους τομείς. Και οι δύο μεθοδολογίες είναι σημαντικές για την ενίσχυση της ασφάλειας ενός οργανισμού, καθεμία προσφέροντας μοναδικά πλεονεκτήματα στη διεξαγωγή \lt pen testing.
\subsubsection{\gt Εργαλεία του \lt Penetration testing}

\hspace*{2em}Όπως αναφέρθηκε και στα παραπάνω κεφάλαια οι ελεγκτές που διεξάγουν τα \lt penetration test \gt χρησιμοποιούν εκτός από μεθοδολογίες διάφορα εργαλεία για να αυτοματοποιήσουν διαδικασίες όπως την εύρεση ευάλωτων σημείων ή ακόμα και την εκμετάλλευσή τους. Κάποια από τα πιο γνωστά εργαλεία είναι το \lt NMap ,  Wireshark , OWASP ZAP , John the Ripper , Metasploit ,SQLmap \gt και \lt Burp Suite. \gtΤα περισσότερα από αυτά είναι \lt open-source \gtκαι σε κάποιες διανομές \lt Linux \gt όπως τα \lt KALI \gt και \lt Parrot OS \gtείναι προ εγκατεστημένα.


\begin{center}
    \textbf{Εργαλεία για την σάρωση πληροφοριών}
\end{center}
    

    Πρόκειται για εργαλεία που χρησιμοποιούνται στην αρχική φάση ενός \lt pen test \gt όπου ελεγκτής συγκετρώνει πληροφορίες για το σύστημα.Η χρήση εργαλείων σε αυτή την φάση είναι ιδιαίτερα χρήσιμη καθώς βοηθούν τον ελεγκτή να κατανοήσει καλύτερα την τοπολογία του δικτύου, τις υπηρεσίες που εκτελούνται σε ένα σύστημα και να εντοπίσει πιθανές εισόδους στο σύστημα.
   Το \lt NMap \gt είναι ένα ευρέως χρησιμοποιούμενο εργαλείο σάρωσης δικτύου που βοηθά στην ανακάλυψη κεντρικών υπολογιστών, υπηρεσιών και ανοιχτών θυρών, παρέχοντας πληροφορίες για τα λειτουργικά συστήματα και τις εκδόσεις υπηρεσιών που εκτελούνται στο δίκτυο.Ένα ακόμα χρήσιμο εργαλείο το \lt Maltego \gt χρησιμοποιείται για  εξόρυξη δεδομένων η οποία βοηθά στην οπτικοποίηση των σχέσεων μεταξύ σημείων δεδομένων, καθιστώντας το ιδιαίτερα χρήσιμο για τη συλλογή πληροφοριών ανοιχτού κώδικα \lt (OSINT).\gtΕπιπλέον, το \lt Google Dorking \gt είναι μια τεχνική που χρησιμοποιεί ειδικούς τελεστές αναζήτησης στο \lt Google \gt για να βρει πληροφορίες που δεν είναι άμεσα εμφανείς σε ιστότοπους αλλά έχουν καταχωρηθεί από τις μηχανές αναζήτησης.

\vspace{1em}
\begin{center}
   \textbf{Εργαλεία για σάρωση ευπαθειών}
\end{center}
    Στις αρχικές φάσεις ενός \lt pen test \gt ο ελεγκτής χρησιμοποιεί εργαλεία για να εντοπίσει ευάλωτα σημεία σε ένα σύστημα.Γνωστό εργαλείο για σάρωση ευπαθειών ικανό να εντοπίζει ένα μεγάλο εύρος ζητημάτων ασφαλείας, όπως γνωστές ευπάθειες, λανθασμένες ρυθμίσεις και μη ενημερωμένο λογισμικό. είναι το \lt Nessus \gt. Το \lt OpenVAS \gt είναι μια  εναλλακτική λύση του \lt Nessus \gt καθώς παρέχει παρόμοια λειτουργικότητα για τον εντοπισμό αδυναμιών ασφάλειας και είναι \lt open-source. 
\vspace{1em}
\begin{center}
    \textbf{Εργαλεία για εκμετάλλευση ευάλωτων σημείων}
\end{center}
    

    Η φάση του \lt exploitation \gt είναι από τις πιο κρίσιμες φάσεις ενός \lt pen test \gt αφού από τα αποτελέσματα αυτής της φάσης θα κριθεί πόσο σημαντική είναι η ευπάθεια που βρέθηκε και πως θα επηρεαστεί το σύστημα σε περίπτωση που κάποιος την εκμεταλλευτεί.Ένα από τα πιο σημαντικά εργαλεία σε αυτή τη φάση είναι το \lt Metasploit Framework \gt,το οποίο έχει την ικανότητά  να προσομοιώνει πολύπλοκες επιθέσεις σε διάφορες πλατφόρμες. Επιτρέπει στους ελεγκτές να εκμεταλλεύονται συστηματικά τα ευάλωτα σημεία, παρέχοντας πολύτιμες πληροφορίες για τη κατάσταση ασφαλείας του στόχου.Άλλο ένα διαδεδομένο εργαλείο είναι το \lt SQLmap \gt , ειδικά σχεδιασμένο για να αυτοματοποιεί τον εντοπισμό και την εκμετάλλευση ευπαθειών τύπου \lt SQL injection \gt σε \lt web \gtεφαρμογές.
\begin{center}
    \textbf{Εργαλεία για \lt web \gtεφαρμογές}
\end{center}
    Για τη διεξαγωγή  \lt web application pen tests\gt, υπάρχουν εργαλεία που διευκολύνουν τη διαδικασία και την καθιστούν πιο αποτελεσματική. Δύο από τα πιο ευρέως χρησιμοποιούμενα είναι το \lt Burp Suite \gtκαι το \lt OWASP ZAP (Zed Attack Proxy).\gtΤο \lt Burp Suite \gtπροσφέρει λειτουργίες όπως \lt proxy server\gt, σαρωτή ευπαθειών και επαναλήπτη\lt (repeater)\gt, επιτρέποντας στους ελεγκτές να παρακολουθούν, να τροποποιούν και να αναλύουν την κυκλοφορία ιστού, να εντοπίζουν ευάλωτα σημεία και να αυτοματοποιούν επαναλαμβανόμενες εργασίες. Από την άλλη, το \lt OWASP ZAP \gt είναι μια \lt open-source \gt εναλλακτική λύση που επίσης παρέχει ισχυρές δυνατότητες σάρωσης για την ανίχνευση ευπαθειών σε \lt web \gt εφαρμογές. Είναι φιλικό προς το χρήστη και υποστηρίζει τόσο αυτοματοποιημένες όσο και μη αυτόματες δοκιμές, καθιστώντας το ιδανικό για αρχάριους και έμπειρους \lt pen testers.\gt
\vspace{1em}
\begin{center}
    \textbf{Εργαλεία για σπάσιμο κωδικών πρόσβασης}
\end{center}
    Το σπάσιμο κωδικού πρόσβασης είναι ένας από τους βασικότερους τρόπους να αξιολογηθεί η ασφάλεια ενός συστήματος. Ο ελεγκτής επιχειρεί να αποκτήσει πρόσβαση ,για παράδειγμα σε λογαριασμό κάποιου άλλου χρήση χωρίς την έγκριση του, αποκωδικοποιόντας τον κατακερματισμένο κωδικό πρόσβασης. Δύο κορυφαία εργαλεία σε αυτόν τον τομέα είναι το \lt John the Ripper \gt και το \lt Hashcat.\gtΤο \lt John the Ripper \gt είναι γνωστό για την ταχύτητα και την ευελιξία που προσφέρει, καθώς λειτουργεί με πολλούς διαφορετικούς τύπους αλγορίθμων κρυπτογράφησης και κατακερματισμού. Χρησιμοποιεί μεθόδους όπως \lt dictionary attacks \gt και \lt brute force attacks \gt για να σπάσει αποτελεσματικά τους κωδικούς πρόσβασης.Το \lt Hashcat\gt, από την άλλη, ξεχωρίζει επειδή για να επιταχύνει τη διαδικασία χρησιμοποιεί την \lt GPU\gt. Αυτό το καθιστά ιδιαίτερα χρήσιμο εργαλείο σε περιπτώσεις περίπλοκων  αλγορίθμων κατακερματισμού που απαιτούν πολλούς πόρους. Υποστηρίζει επίσης πολλούς τύπους κατακερματισμών και προσφέρει προηγμένες μεθόδους επίθεσης, όπως \lt hybrid attacks \gt και \lt rule-based attacks\gt, καθιστώντας το πολύ αποτελεσματικό στην ανάκτηση κωδικών πρόσβασης από διάφορα κρυπτογραφημένα δεδομένα.
 




\subsection{\lt Web \gtεφαρμογές και ασφάλεια}
\subsubsection{Τι είναι μια \lt web \gt εφαρμογή}

Μια \lt web \gt εφαρμογή ή αλλιώς διαδικτυακή εφαρμογή είναι μια εφαρμογή λογισμικού που εκτελείται σε έναν \lt web server \gt και είναι προσβάσιμη μέσω \lt web browser.\gtΣε αντίθεση με τις παραδοσιακές εφαρμογές ,οι οποίες είναι εγκατεστημένες στον υπολογιστή ή σε κάποια κινητή συσκεύη, οι \lt web \gtεφαρμογές είναι προσβάσιμες από οποιαδήποτε συσκευή με σύνδεση στο διαδίκτυο και συμβατό πρόγραμμα περιήγησης(\lt web browser).

\gtΟι \lt web \gt εφαρμογές μπορεί να ποικίλλουν από απλούς ιστότοπους με διαδραστικές λειτουργίες, όπως φόρμες επικοινωνίας, έως πολύπλοκες πλατφόρμες όπως ιστότοποι μέσων κοινωνικής δικτύωσης ή συστήματα \lt e-banking\gt.Χρησιμοποιούν έναν συνδυασμό τεχνολογιών \lt front-end (user interface) \gtκαι \lt back-end (server-side)\gt.

\begin{center}
    \textbf{Πως λειτουργεί μια \lt web \gtεφαρμογή}
\end{center}

Οι \lt web \gtεφαρμογές αποτελούνται από δύο δομικά μέρη,το \lt client \gtκαι το \lt server\gt.Ο \lt client \gt,που είναι συνήθως ένας \lt web browser\gt , αλληλεπιδρά με τον \lt server \gt για την εκτέλεση ενεργειών στην εφαρμογή. Το μέρος του \lt client \gt είναι υπεύθυνο για το \lt user interface \gt μιας εφαρμογής,δηλαδή είναι το κομμάτι με το οποίο αλληλεπιδρά ένας χρήστης. Όταν για παράδειγμα ένας χρήστης κάνει κλικ σε ένα κουμπί ή υποβάλλει μια φόρμα, το \lt front-end(client) \gt στέλνει ένα αίτημα στο \lt back-end(server) \gt ,με τη σειρά του ο \lt server \gt επεξεργάζεται αυτά τα αιτήματα, εκτελεί την απαραίτητη λογική, ανακτά ή ενημερώνει δεδομένα από μια βάση δεδομένων και στη συνέχεια στέλνει μια απάντηση στο \lt front-end.\gtΑυτή η απάντηση μπορεί να είναι οτιδήποτε, από τη φόρτωση μιας νέας σελίδας έως την ενημέρωση ενός μέρους της ιστοσελίδας χωρίς πλήρη επαναφόρτωση.


Καθώς οι \lt web \gt εφαρμογές έχουν ως σκοπό την καλύτερη εμπειρία για τον χρήστη πρέπει να βρουν έναν τρόπο ώστε να διαχειρίζονται όσο πιο αποδοτικά γίνεται τις περιόδους σύνδεσης και την αποθήκευσή δεδομένων/προτιμήσεων ενός χρήστη.Εδώ μπαίνουν στο "παιχνίδι" τα \lt cookies \gt.Τα \lt cookies \gt διαδραματίζουν ιδιαίτερα σημαντικό ρόλο στη διαχείριση περιόδων σύνδεσης και στην αποθήκευση δεδομένων από το μέρος του \lt client\gt. Ένα \lt cookie \gtείναι ένα μικρό κομμάτι δεδομένων που στέλνει ο server στο \lt browser\gt  του \lt client\gt, το οποίο στη συνέχεια αποθηκεύεται τοπικά στη συσκευή του χρήστη. Τα \lt cookies \gt χρησιμοποιούνται για διάφορους σκοπούς, όπως την απομνημόνευση των προτιμήσεων των χρηστών, την διατήρηση της σύνδεσης των χρηστών στην εφαρμογή ή η παρακολούθηση της συμπεριφοράς των χρηστών σε πολλές περιόδους σύνδεσης. Για παράδειγμα, όταν συνδέεστε σε έναν ιστότοπο και επιλέγετε την επιλογή \lt "Remember me"\gt, ρυθμίζεται ένα \lt cookie \gt για να σας κρατά συνδεδεμένους ακόμα και αφού κλείσετε το \lt browser \gt σας. Τα \lt cookies \gt μπορούν επίσης να χρησιμοποιηθούν για αναλυτικά στοιχεία και διαφημίσεις, βοηθώντας τις \lt web \gt εφαρμογές  να προσφέρουν εξατομικευμένες εμπειρίες με βάση τις παλαιότερες προτιμήσεις των χρηστών.

Οι βάσεις δεδομένων αποτελούν ένα από τα σημαντικότερα μέρη μιας \lt web \gtεφαρμογής καθώς προσφέρουν  ένα δομημένο τρόπο αποθήκευσης, ανάκτησης και διαχείρισης δεδομένων. Όταν ένας χρήστης αλληλεπιδρά με μια εφαρμογή—υποβάλλοντας μια φόρμα, πραγματοποιώντας μια αγορά ή δημοσιεύοντας ένα σχόλιο—τα δεδομένα που δημιουργούνται αποθηκεύονται σε μια βάση δεδομένων. Οι βάσεις δεδομένων μπορούν να ταξινομηθούν σε τύπους \lt SQL (Structured Query Language) \gtκαι \lt NoSQL\gt.Οι βάσεις δεδομένων \lt SQL,\gt όπως η \lt MySQL \gtκαι η \lt PostgreSQL\gt, οργανώνουν δεδομένα σε πίνακες με προκαθορισμένα σχήματα, καθιστώντας τις ιδανικές για εφαρμογές που απαιτούν πολύπλοκα \lt querys \gt και σχέσεις μεταξύ των δεδομένων.Στην ουσία αυτές οι βάσεις δεδομένων διασφαλίζουν την ακεραιότητα της αποθήκευσης των δεδομένων μέσω δομημένων σχημάτων και σχεσιακών μοντέλων.

Οι βάσεις δεδομένων \lt NoSQL,\gt όπως η \lt MongoDB\gt, προσφέρουν μια πιο ευέλικτη προσέγγιση, αποθηκεύοντας δεδομένα σε μορφές όπως έγγραφα, ζεύγη \lt key-values\gt ή γραφήματα, καθιστώντας τις κατάλληλες για εφαρμογές που χειρίζονται μεγάλους όγκους μη δομημένων ή ημιδομημένων δεδομένων. Οι βάσεις δεδομένων \lt NoSQL \gtχρησιμοποιούνται συχνά σε καταστάσεις όπου η επεκτασιμότητα και η απόδοση είναι κρίσιμες, όπως σε αναλύσεις σε πραγματικό χρόνο, συστήματα διαχείρισης περιεχομένου και εφαρμογές μεγάλων δεδομένων. Επιτρέπουν πιο ευέλικτη ανάπτυξη και μπορούν να χειριστούν διαφορετικούς τύπους δεδομένων πιο αποτελεσματικά από τις παραδοσιακές βάσεις δεδομένων \lt SQL\gt.



\begin{center}
    \textbf{Τεχνολογίες  που χρησιμοποιούνται σε μια \lt web \gtεφαρμογή}
\end{center}
Οι \lt web \gt εφαρμογές βασίζονται σε μια ποικιλία τεχνολογιών για να λειτουργούν αποτελεσματικά και να είναι εύκολες στην χρήση από τον καθένα. Η ανάπτυξη τους βασίζεται σε ορισμένες βασικές γλώσσες προγραμματισμού , καθεμία από τις οποίες έχει έναν συγκεκριμένο  ρόλο στη δημιουργία τους.Η \lt HTML (HyperText Markup Language) \gtείναι η βασική γλώσσα που χρησιμοποιείται για τη δομή μιας web εφαρμογής, ορίζοντας τα βασικά δομικά στοιχεία μιας ιστοσελίδας, όπως επικεφαλίδες, συνδέσμους, εικόνες και πολυμέσα. Η \lt HTML5\gt, η τελευταία έκδοση της \lt HTML\gt, έχει βελτιώσει σημαντικά την ανάπτυξη ιστοσελίδων, εισάγοντας νέα σημασιολογικά στοιχεία, εργαλεία για την ενσωμάτωση πολυμέσων και \lt API \gt που επιτρέπουν τη δημιουργία πιο δυναμικών και εύχρηστων εφαρμογών. 

Εκτός από την \lt HTML,\gt η \lt CSS (Cascading Style Sheets) \gt  χρησιμοποιείται για το στυλ και την οπτική μορφοποίηση μιας \lt web \gt εφαρμογής, επιτρέποντας στους προγραμματιστές να ελέγχουν τη διάταξη, τα χρώματα, τις γραμματοσειρές και τη συνολική εμφάνιση μιας ιστοσελίδας.Η \lt CSS \gt επιτρέπει τον διαχωρισμό του περιεχομένου από το σχεδιασμό, καθιστώντας ευκολότερη τη διατήρηση και ενημέρωση των οπτικών πτυχών ενός ιστότοπου χωρίς να αλλοιώνεται η υποκείμενη δομή του. Οι προηγμένες λειτουργίες όπως το \lt CSS Grid \gt και το \lt Flexbox \gt έχουν δώσει περαιτέρω τη δυνατότητα στους προγραμματιστές να δημιουργούν σύνθετες διατάξεις που προσαρμόζονται σε διαφορετικά μεγέθη οθόνης ανάλογα με την  συσκεύη, διασφαλίζοντας ότι η ιστοσελίδα θα λειτουργεί ομαλά σε Η/Υ, \lt tablet \gt και \lt smartphone.\gt

Άλλη μια γλώσσα προγραμματισμού που χρησιμοποιείται στην πλειοψηφία των \lt web \gt εφαρμογών ,κυρίως για το \lt back-end \gt κομμάτι, είναι η \lt JavaScript\gt. Η \lt JavaScript \gt επιτρέπει στους προγραμματιστές να δημιουργούν λειτουργίες όπως φόρμες επικύρωσης,\lt animations\gt,διαδραστικούς χάρτες και \lt real-time \gt ενημέρωση της εφαρμογής, βελτιώνοντας σημαντικά την εμπειρία χρήστη. Οι δυνατότητες της \lt JavaScript \gt επεκτείνονται περαιτέρω μέσω βιβλιοθηκών και frameworks όπως τα \lt jQuery, React, Angular \gt και \lt Vue.js\gt, τα οποία απλοποιούν τη διαδικασία δημιουργίας πολύπλοκων \lt user interfaces.\gt

Στην πλευρά του server χρησιμοποιούνται επίσης γλώσσες όπως \lt PHP,Python,Ruby \gt και \lt Java \gt  για να δημιουργούν τη λογική και τη λειτουργικότητα που απαιτούνται για τις \lt web \gtεφαρμογές. Αυτές οι γλώσσες βοηθούν στην επεξεργασία δεδομένων, τη διαχείριση βάσεων δεδομένων και την εκτέλεση πολύπλοκων λειτουργιών. Για παράδειγμα, η \lt PHP \gtχρησιμοποιείται συχνά για την ανάπτυξη ιστοσελίδων,σε συστήματα όπως το \lt WordPress\gt. Η \lt Python\gt, γνωστή για την ευκολία στη χρήση της, χρησιμοποιείται σε \lt frameworks \gtόπως το \lt Django \gtκαι το \lt Flask \gtγια την κατασκευή εφαρμογών. Η \lt Ruby,\gt με το \lt framework Rails\gt, είναι δημοφιλής για την ταχύτητα ανάπτυξης \lt web \gtεφαρμογών, ενώ η \lt Java\gt, με την ευρεία χρήση της σε εταιρικές εφαρμογές, είναι γνωστή για την απόδοση και την ασφάλειά της.

Τέλος ένα ιδιαίτερα σημαντικό χαρακτηριστικό των σύγχρονων \lt web \gt εφαρμογών είναι η δυνατότητά τους να επικοινωνούν και να συνδέονται με άλλες υπηρεσίες λογισμικού. Αυτό γίνεται μέσω των \lt Application Programming Interfaces(APIs)\gt. Τα \lt API \gt επιτρέπουν στις εφαρμογές να στέλνουν και να λαμβάνουν δεδομένα από άλλες υπηρεσίες, διευκολύνοντας πολλές λειτουργίες. Για παράδειγμα, μια ιστοσελίδα μπορεί να χρησιμοποιεί \lt API \gtόπως το \lt Stripe \gtή το \lt PayPal \gtγια να διαχειρίζεται με ασφάλεια τις πληρωμές. Επίσης, είναι συχνό να ενσωματώνονται API από πλατφόρμες κοινωνικής δικτύωσης, όπως το \lt Facebook \gtή το \lt Twitter\gt, για να επιτρέπεται στους χρήστες να συνδέονται,για παράδειγμα μέσω \lt Facebook,\gt ή να μοιράζονται περιεχόμενο μέσα στην εφαρμογή. Επιπλέον, τα \lt API \gtμπορούν να χρησιμοποιηθούν για να εισάγουν δεδομένα από εξωτερικές πηγές, όπως πληροφορίες καιρού ή ειδήσεις, που μπορούν να εμφανιστούν μέσα στην εφαρμογή.

\subsubsection{Ευπάθειες \lt web \gt εφαρμογών}

Οι πιο συχνές αδυναμίες σε μια \lt web \gtεφαρμογή εμφανίζονται στο σχεδιασμό, την υλοποίηση ή τη διαμόρφωση της  εφαρμογής. Αυτά τα ευάλωτα σημεία ενέχουν σημαντικούς κινδύνους, οδηγώντας σε παραβιάσεις δεδομένων ή ακόμη οικονομικές απώλειες στην εφαρμογή.Σε αυτήν την ενότητα, θα αναλυθούν οι 10 πιο σημαντικές ευπάθειες που μπορεί να επηρεάσουν μια \lt web \gtεφαρμογή σύμφωνα με το \lt OWASP(Open Worldwide Application Security Project).\gt

\begin{center}
    \textbf{\lt Broken Access Control}
\end{center}

Ο έλεγχος πρόσβασης(\lt access control) \gt διακυβεύεται όταν γίνεται δυνατή η μη εξουσιοδοτημένη πρόσβαση σε δεδομένα ή λειτουργίες, που συνήθως προστατεύονται από κάποιο \lt user authentication\gt. Η εκμετάλλευση αυτών των ευάλωτων σημείων μπορεί να επιτρέψει στους εισβολείς να πραγματοποιήσουν μη εξουσιοδοτημένες ενέργειες, όπως να κλέψουν ευαίσθητες πληροφοριές, να τροποποίησουν λογαριασμούς χρηστών ή ακομά και να αποκτήσουν πρόσβαση σε λειτουργίες διαχείρισης. Για την αποφυγή τέτοιων παραβιάσεων, είναι σημαντικό να εφαρμόζονται οι αρχές ελάχιστων προνομίων και να επιβάλλεται \lt role-based access control (RBAC).
\begin{center}
    \textbf{\lt Cryptographic Failures}
\end{center}

Οι κρυπτογραφικές αποτυχίες συμβαίνουν συνήθως όταν η κρυπτογράφηση είναι αδύναμη, με αποτέλεσμα την έκθεση προσοπικών δεδομένων. Αυτό μπορεί να οφείλεται στην χρήση πολύ παλιών αλγορίθμων κρυπτογράφησης, κακή διαχείριση κλειδιών ή αποτυχία κρυπτογράφησης ευαίσθητων πληροφοριών συνολικά. Για να αποφευχθεί αυτό, θα πρέπει να χρησιμοποιείται ισχυρή κρυπτογράφηση για την προστασία των δεδομένων τόσο κατά τη μεταφορά όσο και όσο είναι απλά αποθηκευμένα, η διαχείριση των κρυπτογραφικών κλειδιών θα πρέπει να γίνεται με ασφάλεια και δεν θα πρέπει να αποθηκεύονται περιττές ευαίσθητες πληροφορίες ή επιπλέον αντίγραφα.


\begin{center}
    \textbf{\lt Injection}
\end{center}

Ευπάθειες τύπου \lt injection \gtπαρουσιάζονται όταν αποστέλλονται μη αξιόπιστα δεδομένα σε έναν \lt interpreter \gtως μέρος ενός \lt command \gtή \lt query\gt, όπως \lt SQL, NoSQL, OS ή LDAP injections\gt. Οι εισβολείς μπορούν να εκμεταλλευτούν αυτά τα ευάλωτα σημεία για να εκτελέσουν μη εξουσιοδοτημένες εντολές ή να αποκτήσουν πρόσβαση σε δεδομένα. Για την αποφυγή τέτοιων επιθέσεων, συνίσταται η χρήση παραμετροποιημένων \lt query\gt, αποθηκευμένες διαδικασίες και επικύρωση εισόδου.
\newpage
\begin{center}
    \textbf{\lt Insecure Design}
\end{center}

Αυτή η ευπάθεια εμφανίζεται όταν η ασφάλεια δεν έχει προτεραιότητα στην αρχή της ανάπτυξης μιας εφαρμογής. Μπορεί να οδηγήσει σε ζητήματα όπως \lt insecure workflows \gtή έλλειψη βασικών ελέγχων ασφαλείας, δημιουργώντας δυσεύρετα ευπαθή σημεία μόλις η εφαρμογή είναι ενεργή.

\begin{center}
    \textbf{\lt Security Misconfiguration}
\end{center}

Η εσφαλμένη ρύθμιση παραμέτρων ασφαλείας(\lt security misconfiguration) \gt συμβαίνει όταν οι προεπιλεγμένες ρυθμίσεις ασφαλείας δεν ορίζονται ή διατηρούνται σωστά, οδηγώντας στην αύξηση των πιθανών ευάλωτων σημείων με την πάροδο του χρόνου. Αυτό μπορεί να περιλαμβάνει ζητήματα όπως προεπιλεγμένες ρυθμίσεις, ελλιπείς διαμορφώσεις, ανοιχτό χώρο αποθήκευσης στο \lt cloud \gtή υπερβολικά λεπτομερή μηνύματα σφάλματος. Για να αποφευχθεί αυτό, οι οργανισμοί θα πρέπει να δημιουργήσουν ασφαλείς διαμορφώσεις, να αυτοματοποιήσουν την ανάπτυξή τους και να παρακολουθούν και να ελέγχουν τακτικά αυτές τις ρυθμίσεις.

\begin{center}
    \textbf{\lt Vulnerable and Outdated Components}
\end{center}

Η χρήση \lt components \gtμε γνωστές αδυναμίες θέτει σε κίνδυνο την ασφάλεια ολόκληρης της εφαρμογής.Αυτό περιλαμβάνει παλιές βιβλιοθήκες, \lt frameworks \gtκαι άλλα \lt components \gtτου λογισμικού.Για τη διαχείριση αυτών των κινδύνων, είναι απαραίτητη η τακτική ενημέρωση των compoments που χρησιμοποιούνται στην εφαρμογή,επίσης συχνά \lt security tests \gtβοηθούν στον εντοπισμό αυτών των αδυναμιών.

\begin{center}
    \textbf{\lt  Identification and Authentication Failures}
\end{center}

Αυτά τα περιστατικά προέρχονται από αποτυχίες στην αναγνώριση και τον έλεγχο ταυτότητας χρήστη, όπως κακή διαχείριση κωδικού πρόσβασης, αδύναμες πρακτικές ασφάλειας, έλλειψη \lt multi-factor authentication (MFA) \gtή προβλήματα διαχείρισης περιόδου λειτουργίας. Για να αποφευχθεί η επίθεση σε αυτά τα σημεία, είναι σημαντικό να υπάρχουν ισχυροί έλεγχοι ταυτότητας των χρηστών και να υπάρχει σωστή διαχείρισή των περιόδων σύνδεσης του κάθε χρήστη.

\begin{center}
    \textbf{\lt Software and Data Integrity Failures}
\end{center}

Αυτή η κατηγορία αντιμετωπίζει τους κινδύνους που σχετίζονται με τις ενημερώσεις λογισμικού και την ακεραιότητα των δεδομένων. Για παράδειγμα, ένας κακόβουλος χρήστης μπορεί να εκμεταλλευτεί μη ασφαλείς μηχανισμούς ενημέρωσης. Για τον μετριασμό αυτών των κινδύνων, προτείνεται η χρήση ασφαλών διαδικασιών ενημέρωσης του συστήματος, η χρήση κρυπτογραφικών υπογραφών για την επικύρωση της ακεραιότητας του λογισμικού, καθώς και η ενσωμάτωση ελέγχων ασφαλείας σε αγωγούς \lt CI/CD.

\begin{center}
    \textbf{\lt Security Logging and Monitoring Failures}
\end{center}

Η κακή καταγραφή και παρακολούθηση μπορεί να δυσκολέψει τον εντοπισμό ευάλωτων σημείων.Συχνά προβλήματα αυτής της περίπτωσης περιλαμβάνουν την έλλειψη καταγραφής αρχείων, την μη ασφαλή αποθήκευση τους (όπως η διατήρηση διευθύνσεων \lt IP \gtσε απλό κείμενο) και τη δυσκολία εντοπισμού παραβάσεων. Η αποτελεσματική ασφάλεια βασίζεται στην ύπαρξη υψηλής ποιότητας αρχείων καταγραφής, στην παρακολούθηση σε πραγματικό χρόνο για άμεση αντίδραση σε ειδοποιήσεις και σε ένα ισχυρό σύστημα ανάλυσης των αρχείων καταγραφής.

\begin{center}
    \textbf{\lt Server-Side Request Forgery (SSRF)}
\end{center}

Το \lt SSRF (Server-Side Request Forgery) \gtείναι μια ευπάθεια όπου ένας εισβολέας μπορεί να 'ξεγελάσει' τον \lt server \gtώστε να κάνει αυθαίρετα αιτήματα σε άλλα συστήματα, εκθέτοντας πιθανώς εσωτερικές υπηρεσίες και δεδομένα. Για προστασία από επιθέσεις \lt SSRF \gtβοηθάει η περιορισμένη πρόσβαση σε εξωτερικά δίκτυα μόνο όταν είναι απαραίτητη. Επιπλέον, ο οργανισμός πρέπει να διατηρεί το εσωτερικό του δίκτυο καλά τμηματοποιημένο από τα εξωτερικά δίκτυα.

\subsection{\lt SQL Injection}
\subsubsection{Τι είναι μια επίθεση \lt SQL injection}

\hspace*{2em}Μια επίθεση \lt SQL injection \gtείναι μια επίθεση κατά την οποία  ο επιτηθέμενος εκμεταλλεύεται ευπάθειες που υπάρχουν στα \lt query \gt της βάσης δεδομένων μιας \lt web \gtεφαρμογής εισάγοντας κακόβουλο \lt SQL \gtκώδικα σε πεδία εισόδου ή παραμέτρους. Αυτή η επίθεση -αν πετύχει - μπορεί να οδηγήσει σε μη εξουσιοδοτημένη πρόσβαση, διαρροή δεδομένων ή ακόμα και πλήρη έλεγχο της βάσης δεδομένων.Η επίθεση \lt SQL injection \gt ,αν και είναι μια από τις πιο απλές και παλιές μορφές επιθέσεις παραμένει ιδιαίτερα επικίνδυνη για πολλές εφαρμογές λόγω κακής επικύρωσης εισόδου και λόγω 'κακών' \lt query\gt. Ένα κλασικό παράδειγμα επίθεσης \lt SQL injection \gt περιλαμβάνει μια φόρμα σύνδεσης όπου ένας εισβολέας εισάγει \lt «' OR '1'='1'» \gtστο πεδίο κωδικού πρόσβασης, μετατρέποντας το \lt SQL query \gtσε μια συνθήκη που αξιολογείται πάντα ως αληθής, παρέχοντας ενδεχομένως μη εξουσιοδοτημένη πρόσβαση:
\lt
\begin{lstlisting}
SELECT * FROM users WHERE username = 'admin' AND password = '' OR '1'='1';
\end{lstlisting}



Αφού εξηγήσαμε τι είναι μια επίθεση SQL injection, ας δούμε μερικά πραγματικά παραδείγματα όπου τέτοιες επιθέσεις είχαν σημαντικές επιπτώσεις.Στην επίθεση \lt GhostShell\gt, η ομάδα \lt APT Team GhostShell \gtστόχευσε 53 πανεπιστήμια χρησιμοποιώντας \lt SQL injection, \gtκλέβοντας και δημοσιεύοντας 36.000 προσωπικά αρχεία που ανήκαν σε φοιτητές, καθηγητές και προσωπικό. Ένα άλλο παράδειγμα αφορά τη ομάδα \lt RedHack\gt, η οποία χρησιμοποίησε \lt SQL injection \gtγια να παραβιάσει έναν ιστότοπο της τουρκικής κυβέρνησης, διαγράφοντας χρέη προς κυβερνητικές υπηρεσίες. Η παραβίαση του \lt 7-Eleven \gtείναι ένα ακόμη σημαντικό παράδειγμα όπου οι εισβολείς χρησιμοποίησαν \lt SQL injection \gtγια να διεισδύσουν σε εταιρικά συστήματα κλέβοντας 130 εκατομμύρια αριθμούς πιστωτικών καρτών. Τελευταίο παράδειγμα είναι η παραβίαση του \lt HBGary,\gtπου συνέβη από άτομα  που συνδέονται με την ομάδα ακτιβιστών \lt Anonymous,\gtοι οποίοι χρησιμοποίησαν \lt SQL injection \gtγια να κατεβάσουν τον ιστότοπο της  ασφάλειας της εταιρείας. Αυτή η επίθεση έγινε ως αντίποινα για τον Διευθύνοντα Σύμβουλο της \lt HBGary \gtπου δημοσίευσε ότι είχε ταυτοποιήσει μέλη των \lt Anonymous.\gt


\subsubsection{Τύποι \lt SQL injection}
\hspace*{1em}Υπάρχουν αρκετοί διαφορετικοί τύποι \lt SQL injection\gt, οι οποίοι θα αναλυθούν παρακάτω.
\begin{center}
    \textbf{\lt Union-based SQL Injection}
\end{center}

Πρόκειται για την πιο συνηθισμένη μορφή \lt SQL injection \gt κατά τηνοποία ο εισβολέας εκμεταλλεύεται τον τελεστή \lt UNION \gtτης \lt SQL \gtγια να συνδυάσει τα αποτελέσματα δύο ή περισσότερων εντολών \lt SELECT.\gtΑυτό επιτρέπει στον εισβολέα να ανακτήσει δεδομένα από πολλούς πίνακες της βάσης δεδομένων με την χρήση ενός \lt query\gt.Αν η εισαγωγή του \lt query \gtείναι επιτυχής,ο εισβολέας μπορεί να αποκτήσει μη εξουσιοδοτημένη πρόσβαση σε ευαίσθητα προσωπικά δεδομένα δεδομένα που προστατεύονται διαφορετικά, όπως για παράδειγμα \lt user credentials\gt.

\begin{center}
    \textbf{\lt Error-Based Injection}
\end{center}

Σε αυτή την περίπτωση ο εισβολέας εκμεταλλεύεται τα μηνύματα σφάλματος που δημιουργούνται από μια βάση δεδομένων όταν αποτυγχάνει να εκτελεστεί ένα \lt query\gt.Αυτή η μέθοδος είναι ιδιαίτερα αποτελεσματική έναντι των \lt MS-SQL Servers.\gtΟυσιαστικά ο εισβολέας προκαλεί την εφαρμογή να εκτελέσει \lt queries \gtπου θα παράγουν σφάλμα και μέσω του μηνύματος που στέλνει το σφάλμα μπορεί να αποκαλύψει ακούσια πληροφορίες σχετικά με τη δομή της βάσης δεδομένων ή ακόμη και να περιλαμβάνει τα δεδομένα που ζητούνται από τον εισβολέα.

\begin{center}
    \textbf{\lt Blind SQL injection}
\end{center}

Η \lt blind SQL injection \gtείναι η πιο δύσκολη μορφή \lt injection \gtεπειδή δεν παράγει άμεσα μηνύματα σφάλματος ή εξόδους από τα οποία ο εισβολέας μπορεί να λάβει πληροφορίες. Αντίθετα, βασίζεται σε \lt TRUE \gtή \lt FALSE \gtαπαντήσεις από τη βάση δεδομένων για να λάβει πληροφορίες. Αυτό στην συνέχεια μπορεί  να χωριστεί σε \lt boolean-based SQL injection\gt, όπου ο εισβολέας χρησιμοποιεί συγκεκριμένα \lt SQL statements \gtκαι εξάγει την πληροφορία που χρειάζεται για την βάση ανάλογα με την τιμή που λαμβάνει \lt (TRUE/FALSE)  \gtκαι σε \lt time-based SQL injection\gt, όπου ο εισβολέας μετρά το χρόνο που απαιτείται για την εκτέλεση ενός \lt query \gtγια να συμπεράνει εάν πληρούνται ορισμένες προϋποθέσεις.


\newpage

Άλλος ένας τρόπος με τον οποίο μπορούν να ταξινομηθούν είναι με βάση την μέθοδο που χρησιμοποιεί ο εισβολέας για να εισάγει τα δεδομένα.

\begin{center}
    \textbf{\lt SQL Injection Based on User Input}
\end{center}

Αυτός ο τύπος επίθεσης είναι εφικτός όταν η εφαρμογή ενσωματώνει δεδομένα που προέρχονται από τους χρήστες σε \lt queries \gtτης βάσης χωρίς κάποιον έλεγχο ότι αυτά είναι ασφαλή.Είναι η πιο απλή μέθοδος, καθώς τα πεδία εισαγωγής είναι από τους πιο συνηθισμένους τρόπους να εισάγει ο εισβολέας κακόβουλο \lt SQL κώδικα\gt.

\begin{center}
    \textbf{\lt SQL Injection Based on Cookies}
\end{center}


Η επίθεση αυτή περιλαμβάνει τον χειρισμό των δεδομένων που είναι αποθηκευμένα στα cookies ενός χρήστη, τα οποία στη συνέχεια αποστέλλονται στον \lt web server \gtκαι είναι πιθανό να  χρησιμοποιηθούν σε \lt queries \gt της βάσης δεοδμένων της εφαρμογής.Αυτή η περίπτωση είναι δυνατή μόνο αν τα \lt cookies \gtδεν έχουν επικυρωθεί σωστά.
\begin{center}
    \textbf{\lt SQL Injection Based on HTTP}
\end{center}

Στο \lt injection \gtαυτό ο εισβολεάς αξιοποιεί το γεγονός ότι οι \lt web \gtεφαρμογές συχνά δέχονται δεδομένα από \lt HTTP headers\gt, όπως \lt User-Agent \gtή \lt Referer.\gtΟ εισβολέας μπορεί να τροποποιήσει τα \lt headers \gtγια να εισάγουν κώδικα \lt SQL \gtσε \lt queries \gt που εκτελούνται από την εφαρμογή με την χρήση αυτών των \lt header.\gt

\begin{center}
    \textbf{\lt Second-Order SQL Injection}
\end{center}

Πρόκειται για μια ιδιαίτερα ύπουλη μορφή επίθεσης καθώς ο επιτιθέμενος εισάγει κακόβουλου κώδικα που δεν εκτελείται αμέσως. Αντίθετα, ο κώδικας που εισάγεται αποθηκεύεται στη βάση δεδομένων και εκτελείται αργότερα όταν ανακτηθεί για την εκτέλεση κάποιας λειτουργίας. Αυτός ο τύπος \lt injection \gtείναι πολύ δύσκολο να εντοπιστεί επειδή ο κώδικας δεν φαίνεται κάπου μέχρι να ενεργοποιηθεί από μια συγκεκριμένη ενέργεια ή \lt query\gt.
